\chapter{Introduction}
\label{introduction}

\section{Background}
\label{introduction:background}
The Quantum Internet is a new internet that uses quantum entanglement and is expected to be a technology that will make up for the weaknesses of the current Internet.
Specifically, more robust cryptographic algorithms, faster solving of consensus problems and increased computational power through distributed quantum computation. 
However, the Quantum Internet has not yet reached the practical stage due to many hardware limitations. 
Nevertheless, many cryptographic algorithms and protocols are still being proposed so research activities on the Quantum Internet are flourishing.

In implementing the Quantum Internet, it is necessary to test the network. 
This is done to make sure that the protocols, etc. work properly for user use. 
In order to test the network, we need traffic data. 
To prepare traffic data, you need to use real data or generate it stochastically.
However, since the Quantum Internet is not yet in practical use, it is not possible to use real data.
Therefore, it needs to be generated stochastically.
In the current Internet, traffic data is generated from the gravity model and the maximum entropy model, etc.
The purpose of this paper is to establish a method of traffic data generation for the Quantum Internet.

\section{Research Contribution}
\label{introduction:research_contribution}
The main contribution of this project is the establishment of a basis for traffic matrix generation in the Quantum Internet.
The traffic matrix generation method used in the current Internet has been applied to the Quantum Internet.
Thereby, implemented traffic data generation for a Quantum Internet simulator to help simulate the Quantum Internet.

\section{Thesis Structure}
\label{introduction:thesis_structure}
This thesis is constructed as follows.

Chapter 2 describes the fundamentals of quantum information and the Quantum Internet.
In chapter 3, I will introduce the knowledge of the traffic in classical communications and findings from previous studies.
Chapter 4 details the problem definition of this research and the methods used to solve the problem.
Chapter 5 describes how the method was implemented in a Quantum Internet simulator with actual code.
Chapter 6 describes the evaluation of this research. It describes how to test the implemented functions.
Chapter 7 provides a summary of this research and describes the future development of the research.


%%% Local Variables:
%%% mode: japanese-latex
%%% TeX-master: "../thesis"
%%% End:
