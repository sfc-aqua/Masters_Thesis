Abstract of Bachelor's Thesis - Academic Year 2021
\begin{center}
\begin{large}
\begin{tabular}{|p{0.97\linewidth}|}
    \hline
      \etitle \\
    \hline
\end{tabular}
\end{large}
\end{center}

~ \\
The Quantum Internet is a new Internet that uses quantum entanglement, and is expected to be a technology that will make up for the weaknesses of the current Internet.
It is expected to make up for the weaknesses of the current Internet, such as stronger cryptographic algorithms, faster resolution of consensus problems, and increased computing power through distributed quantum computation.
However, practical applications are still far away and are still in the experimental stage.

One of the necessary elements for practical use is traffic testing.
This is a test of the network, to make sure that protocols and other aspects work correctly when users use them.
For traffic testing, we need traffic data, which can be either real data or generated stochastically.
In the case of Quantum Internet, real data does not exist, so we need to generate data.

In this project, I defined "traffic" in the Quantum Internet, and applied the traffic data generation using the gravity model, which is the mainstream in the current Internet, to the Quantum Internet.
I also implemented this model in QUISP, a Quantum Internet simulator ran on OMNeT++, and tested it.
In this way, It have been shown the basis for traffic testing in the Quantum Internet.
~ \\
Keywords : \\
\underline{1. Quantum Internet},
\underline{2. Traffic Matrix},
\underline{3. Gravity Model},
\underline{4. Traffic Generation}
\begin{flushright}
\edept \\
\eauthor
\end{flushright}
