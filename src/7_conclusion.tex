\chapter{Conclusion}
\label{conclusion}
\section{Conclusion}
In this project, I defined "traffic" in the Quantum Internet, and applied the traffic data generation using the gravity model, which is the mainstream in the current Internet, to the Quantum Internet.
I also implemented this model in QUISP, a Quantum Internet simulator ran on OMNeT++, and tested it.
I also evaluated the traffic matrices generated using several measures of errors.
In this way, It have been shown the basis for traffic testing in the Quantum Internet.

As for future work, the first step is to conduct traffic tests using the generated traffic matrix to verify its effectiveness.
It is currently not possible to test this due to the simulator's functionality, but we should add the missing functionality and do a traffic test to check the behavior.
The next step is to implement a model other than the gravity model and compare each model.
The gravity model is a very simple model. 
Simplicity is a good thing, but too much simplicity is not so good.
Other models have already been proposed for generating traffic matrices in the current Internet, so it would be worthwhile to try them in the Quantum Internet.
%%% Local Variables:
%%% mode: japanese-latex
%%% TeX-master: "../thesis"
%%% End:
